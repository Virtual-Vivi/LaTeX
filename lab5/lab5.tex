\documentclass{article}

\usepackage{fontspec}
\setmainfont{CMU Serif}
\usepackage[english, ukrainian]{babel}
\usepackage{microtype}
\usepackage{amsmath}
\usepackage{xcolor}

\newcounter{ExampleCounter}
\numberwithin{ExampleCounter}{section}
\numberwithin{equation}{ExampleCounter}

\renewcommand{\theequation}{\theExampleCounter--\arabic{equation}}

\NewDocumentEnvironment{Example}{}
{\par\hrulefill\par
 \refstepcounter{ExampleCounter}
 \textbf{Приклад}~\theExampleCounter:
 \color{red}}
{\color{black}\par\hrulefill}

\title{Енергія}
\author{}
\date{}

\begin{document}
\maketitle

%% --------------------------------------------------------
\section{Теорія відносності}
%% --------------------------------------------------------

У спеціальній теорії відносності енергія тіла розглядається як фундаментальна величина, яка тісно
пов'язана з його масою та рухом. Одним із ключових принципів цієї теорії є еквівалентність маси та
енергії, яка показує, що маса може перетворюватися на енергію і навпаки. Цей принцип став основою для
розуміння таких явищ, як ядерна енергія та процеси, що відбуваються в надрах зірок.

\begin{Example}\label{ex:einstein}
	Одним із найвідоміших рівнянь у фізиці є рівняння Ейнштейна, яке описує еквівалентність маси та
	енергії:

	\begin{equation}\label{eq:einstein}
		E_0 = mc^2
	\end{equation}

	Це рівняння показує, що енергія спокою ($E_0$) тіла дорівнює його масі, помноженій на квадрат
	швидкості світла.

    Частинки які рухаються зі швидкість світла --- фотони --- маси спокою не мають. Для них:
    \begin{equation}
        E_0 = 0
    \end{equation}

\end{Example}

Крім того, теорія відносності вказує на те, що енергія тіла залежить не лише від його маси, але й від
швидкості його руху. Чим швидше рухається тіло, тим більшою стає його енергія. Це призводить до того,
що для розгону тіла до швидкостей, близьких до швидкості світла, потрібна величезна кількість
енергії. Така залежність енергії від швидкості має важливе значення для розуміння поведінки частинок
у прискорювачах та космічних об'єктів.



\begin{Example}\label{ex:einstein2}
	Для тіла, що рухається зі швидкістю $v$ його енергія дається виразом:
	\begin{equation}\label{eq:einstein2}
		E = \gamma mc^2
	\end{equation}
де $\gamma = \dfrac1{\sqrt{1-\dfrac{v^2}{c^2}}}$.
\end{Example}

%% --------------------------------------------------------
\section{Квантова механіка}
%% --------------------------------------------------------


У квантовій механіці енергія набуває цілком нового значення, відмінного від класичного підходу. На
мікроскопічному рівні енергія частинок, таких як електрони чи фотони, не є неперервною величиною.
Навпаки, вона квантується, тобто може приймати лише певні дискретні значення. Це означає, що енергія
частинок змінюється стрибкоподібно, а не плавно, як у класичній фізиці.

\begin{Example}\label{ex:plank}
	У квантовій механіці енергія частинки, наприклад, фотона, пов'язана з її частотою :

	\begin{equation}\label{eq:plank}
		E = \hbar\omega
	\end{equation}

	Це рівняння описує енергію квантового стану та є фундаментальним для розуміння поведінки
	частинок на мікроскопічному рівні. Воно також показує, що енергія квантована, тобто може
	приймати лише певні дискретні значення.

	Порівнюючи це рівняння з рівнянням Ейнштейна \eqref{eq:einstein}, можна побачити, що обидва
	вони описують енергію, але в різних контекстах: перше — у макросвіті, а друге — у мікросвіті.
\end{Example}

%% --------------------------------------------------------
\section{Висновок}
%% --------------------------------------------------------

Рівняння з прикладу~\ref{ex:einstein}: \eqref{eq:einstein} і прикладу~\ref{ex:einstein2}:
\eqref{eq:einstein2} та рівняння \ref{eq:plank} з
прикладу~\ref{ex:plank}, є ключовими для сучасної фізики. Вони
демонструють, як енергія може бути представлена через різні фізичні величини: масу та частоту.
Ці рівняння підкреслюють єдність фізичних законів, які описують як великі, так і малі об'єкти у
Всесвіті.



\end{document}
