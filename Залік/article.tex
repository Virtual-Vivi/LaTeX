%%% --------------------------------------------------------
%%> \section{Директиви компілятора}
%%% --------------------------------------------------------

% !TeX program = lualatex


\documentclass[]{iptconf}

\usepackage[utf8]{inputenc}
\usepackage[backend=biber]{biblatex}



%%% --------------------------------------------------------
%%> \section{Реєстраційна форма}
%%% --------------------------------------------------------





\regform{
    fullname = {Чінь Ву Чіеу Ві},          % Повне ім'я доповідача (перший автор)
    birthday = {29.07.2004},                    % Дата народження доповідача
    position = {студент},                      % Посада доповідача
    phone = {+380990416183},                    % Телефонний номер доповідача
    authoremail = {trinhvutrieuvy@gmail.com},          % Email доповідача
    confsection = {Математичне моделювання та аналіз  даних}, % Секція конференції,
    copynum = {0},                              % Замовлена число друкованого збірника
    needliving = {ні},                          % Потреба в житлі (Ні/Хостел/Готель/інше)
    needinvitanion = {ні}                      % Чи потрібне запрошення на конференцію?
}





%%% --------------------------------------------------------
%%> \section{Використані пакети}
%%% --------------------------------------------------------





\usepackage[most]{tcolorbox}
\usepackage{tabularray}
\UseTblrLibrary{booktabs}
\usepackage{mathtools}
\usepackage{dsfont}
\usepackage{mathrsfs}
\usepackage{wrapfig}
\usepackage{xurl}
\usepackage[version=4]{mhchem}
\usepackage{forest}
\usepackage{tikz}
\usepackage{pgfplots}
\pgfplotsset{compat=1.18}
\usepackage{listings}
\usetikzlibrary{shadows,arrows.meta}





%%% --------------------------------------------------------
%%> \section{Файл бібліографії}
%%% --------------------------------------------------------





%% Змініть ім'я файлу бібліографі на ваш.
%% Краще, щоб його назва була така ж сама,
%% як у вашого .tex-файлу

\addbibresource{Bibliography.bib}





%%% --------------------------------------------------------
%%> \section{Команди користувача}
%%% --------------------------------------------------------




%%% --------------------------------------------------------
%%> \section{Заголовок статті}
%%% --------------------------------------------------------





\title{Моделювання впливу забруднення річки Дніпро на популяцію риб: аналіз даних та прогнозування}





%%% --------------------------------------------------------
%%> \section{Автори}
%%% --------------------------------------------------------





%%Якщо бажаєте, введіть
% e-mail автора в квадратних дужках:
\author{В.~Ч.~В.~Чінь}{1}

%%% --------------------------------------------------------
%%> \section{Установи}
%%% --------------------------------------------------------





%% Тут введіть установув якій працює, або навчається перший автор.
%% Введіть \ipt якщо автор навчається, або працює в НТУУ "КПІ"
\affiliation{\ipt}{1}

%%% --------------------------------------------------------
%%> \section{УДК та PACS}
%%% --------------------------------------------------------


%\pacs{ }
\udc{501}


%%% --------------------------------------------------------
%%> \section{Анотація до статті}
%%% --------------------------------------------------------

\abstract{
	У цій роботі досліджується вплив рівня забруднення річки Дніпро на популяцію риб у період з 2013 по 2025 роки. Використовуючи математичні методи аналізу даних, виконується обробка та відновлення пропущених значень щодо концентрації забрудників та обсягів вилову риби.
	Основна частина дослідження присвячена побудові математичних моделей, що описують залежність чисельності риб від рівня забруднення. Для цього використовуються статистичні методи, регресійний аналіз, машинне навчання (методи прогнозування часових рядів), а також диференціальні рівняння для моделювання динаміки екосистеми. На основі отриманих моделей розглядаються різні сценарії змін екологічної ситуації, зокрема прогнозується вплив можливих екологічних заходів на біорізноманіття річки.
	Результати дослідження можуть бути корисними для оцінки довгострокових наслідків забруднення водних ресурсів та розробки стратегій збереження екосистеми Дніпра.
}


%%% --------------------------------------------------------
%%> \section{Ключові слова}
%%% --------------------------------------------------------





\keywords{математичне моделювання, забруднення води, річка Дніпро, популяція риб, статистичний аналіз, диференціальні рівняння, прогнозування}


\begin{document}


%%% --------------------------------------------------------
%%> \section{Мова статті}
%%% --------------------------------------------------------





\PaperLanguage{ukrainian} %





%%% --------------------------------------------------------
\section*{Вступ}
%%% --------------------------------------------------------





Річка Дніпро є однією з найважливіших водних артерій України, що забезпечує водопостачання, сільське господарство, промисловість та рибальство. Однак зростаючий антропогенний тиск та погіршення екологічного стану річки викликають занепокоєння. Забруднення води важкими металами, нітратами, фосфатами та мікропластиком сприяє деградації водних екосистем та скороченню популяцій риб \cite{Gryb}.

Одним із підходів до оцінки впливу забруднення на водні біоценози є математичне моделювання, яке дозволяє аналізувати наслідки змін екологічних факторів та прогнозувати майбутні сценарії.





%%% --------------------------------------------------------
\section{Забруднення Дніпра}
%%% --------------------------------------------------------




%%% --------------------------------------------------------
\subsubsection*{Основні джерела та наслідки}
%%% --------------------------------------------------------



Забруднення води у Дніпрі відбувається через скиди промислових і побутових стоків, сільськогосподарську діяльність, а також випадання атмосферних забруднень. Основними забруднювачами є:

\begin{itemize}
	\item Токсичні метали (кадмій, ртуть, свинець) -- накопичуються в донних відкладеннях та потрапляють у харчовий ланцюг риб.
	\item Нітрати та фосфати -- спричиняють цвітіння води, що зменшує рівень кисню у воді та призводить до загибелі риб.
	\item Органічні сполуки та мікропластик -- негативно впливають на репродуктивну функцію водних організмів.
\end{itemize}

Дослідження \cite{Mudrak} показало, що евтрофікація призводить до зменшення видового різноманіття та чисельності популяцій риб.




%%% --------------------------------------------------------
\subsubsection*{Вплив забруднення на популяцію риб}
%%% --------------------------------------------------------


За даними Державного агентства водних ресурсів України (2021), протягом останнього десятиліття спостерігається скорочення вилову риби, що може бути непрямим свідченням змін у популяціях. Деякі види, такі як судак і осетер, стали рідкісними через деградацію місць нересту та забруднення.

Забруднення водного середовища впливає на:
\begin{itemize}
	\item Виживаність мальків та їхній ріст,
	\item Зміну харчової поведінки риб,
	\item Біоакумуляцію токсичних речовин у тканинах риб.
\end{itemize}

Для більш точної оцінки впливу екологічних факторів використовуються математичні моделі.


%%% --------------------------------------------------------
\section{Методи математичного моделювання екологічних процесів}
%%% --------------------------------------------------------

У прикладній математиці екологічні процеси моделюються за допомогою:
\begin{itemize}
	\item Динамічних систем -- описують зміну популяції риб у часі під впливом забруднення.
	\item Нейронних мереж -- використовуються для прогнозування сценаріїв забруднення.
	 \item Стохастичних моделей -- враховують випадкові коливання у рівнях забруднення та чисельності риб.
\end{itemize}

Такі методи дозволяють оцінити наслідки зміни концентрації забрудників у воді та змоделювати можливі сценарії екологічної ситуації.


%%% --------------------------------------------------------
\section*{Висновки}
%%% --------------------------------------------------------


Огляд літератури показав, що проблема забруднення Дніпра є критичною для його екосистеми та біорізноманіття. Антропогенний вплив призводить до зниження популяцій риб, зміни видового складу та накопичення токсичних речовин.

Математичне моделювання дозволяє аналізувати вплив забруднення на водні біоценози та прогнозувати наслідки змін екологічної політики. Подальші дослідження можуть бути спрямовані на побудову моделі, що дозволить оцінити потенційні сценарії покращення ситуації.

\end{document}